\section{Materiais e Métodos}
\label{sec:meteriais}



\subsection{Exemplo de Listing}

\lstinputlisting[language=Java,label=alg:code,caption=Code Example]{Code/code.java}


\subsection{Exemplo de Algorithm}

\begin{algorithm}
\begin{scriptsize}
\begin{lstlisting}
void printOwing (double amount){
	print Banner();
	//print details
    System.out.println("name: " + _name);
    System.out.println("amout: " + amount);
}
\end{lstlisting}
\end{scriptsize}

\caption{Código fonte em Java}
\label{alg:codigo_java}

\end{algorithm}




\subsection{Exemplo de Cronograma}

Esta seção apresenta um exemplo de tabela para construção de cronograma.

\begin{table}[!htb]
\caption{Distribuição de tarefas por datas}
\label{tab:datas}
\resizebox{\textwidth}{!}{%
\begin{tabular}{l|ccccccc|}
\cline{2-8}
                                                                                & \multicolumn{7}{c|}{\textbf{2025}}                                                                                                                                              \\ \cline{2-8} 
                                                                                & \multicolumn{2}{c|}{\textbf{fevereiro}}             & \multicolumn{2}{c|}{\textbf{março}}                 & \multicolumn{2}{c|}{\textbf{abril}}                 & \textbf{maio} \\ \hline
\multicolumn{1}{|l|}{\textbf{Tarefas}}                                          & \multicolumn{1}{c|}{1ªQ} & \multicolumn{1}{c|}{2ªQ} & \multicolumn{1}{c|}{1ªQ} & \multicolumn{1}{c|}{2ªQ} & \multicolumn{1}{c|}{1ªQ} & \multicolumn{1}{c|}{2ªQ} & 1ªQ           \\ \hline
\multicolumn{1}{|l|}{Desenvolvimento do \textit{script} para coleta dos repositórios}    & \multicolumn{1}{c|}{X}   & \multicolumn{1}{c|}{}    & \multicolumn{1}{c|}{}    & \multicolumn{1}{c|}{}    & \multicolumn{1}{c|}{}    & \multicolumn{1}{c|}{}    &               \\ \hline
\multicolumn{1}{|l|}{Coleta de dados dos repositórios}                          & \multicolumn{1}{c|}{}    & \multicolumn{1}{c|}{X}   & \multicolumn{1}{c|}{}    & \multicolumn{1}{c|}{}    & \multicolumn{1}{c|}{}    & \multicolumn{1}{c|}{}    &               \\ \hline
\multicolumn{1}{|l|}{\textit{Desenvolvimento da solução}} & \multicolumn{1}{c|}{}    & \multicolumn{1}{c|}{X}   & \multicolumn{1}{c|}{}    & \multicolumn{1}{c|}{}    & \multicolumn{1}{c|}{}    & \multicolumn{1}{c|}{}    &               \\ \hline
\multicolumn{1}{|l|}{Coleta de dados}                                & \multicolumn{1}{c|}{}    & \multicolumn{1}{c|}{}    & \multicolumn{1}{c|}{X}    & \multicolumn{1}{c|}{X}   & \multicolumn{1}{c|}{}   & \multicolumn{1}{c|}{}    &               \\ \hline
\multicolumn{1}{|l|}{Aplicação do Método}                           & \multicolumn{1}{c|}{}    & \multicolumn{1}{c|}{}    & \multicolumn{1}{c|}{X}    & \multicolumn{1}{c|}{X}   & \multicolumn{1}{c|}{X}   & \multicolumn{1}{c|}{}    &               \\ \hline
\multicolumn{1}{|l|}{Geração dos painéis}           & \multicolumn{1}{c|}{}    & \multicolumn{1}{c|}{}    & \multicolumn{1}{c|}{}    & \multicolumn{1}{c|}{}    & \multicolumn{1}{c|}{}    & \multicolumn{1}{c|}{X}   &               \\ \hline
\multicolumn{1}{|l|}{Discussão e avaliação dos resultados}                      & \multicolumn{1}{c|}{}    & \multicolumn{1}{c|}{}    & \multicolumn{1}{c|}{}    & \multicolumn{1}{c|}{}    & \multicolumn{1}{c|}{}    & \multicolumn{1}{c|}{}    & X             \\ \hline
\end{tabular}%
}
\end{table} 



\subsection{References}

Bibliographic references must be unambiguous and uniform.  We recommend giving
the author names references in brackets, e.g. \cite{knuth:84},
\cite{boulic:91}, and \cite{smith:99}.

Cormen et al. (2016)\nocite{cormen2002algoritmos} representa uma citação direta.



The references must be listed using 12 point font size, with 6 points of space
before each reference. The first line of each reference should not be
indented, while the subsequent should be indented by 0.5 cm.