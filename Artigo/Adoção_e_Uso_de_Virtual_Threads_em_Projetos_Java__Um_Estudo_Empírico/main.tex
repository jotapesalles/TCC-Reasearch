%%%%%%%%%%%%%%%%%%%%%%%%%%%%%%%%%%%%%%%%%%%%%%%%%%%%%%%%%%%%%%%%%%%%%%
% How to use writeLaTeX: 
%
% You edit the source code here on the left, and the preview on the
% right shows you the result within a few seconds.
%
% Bookmark this page and share the URL with your co-authors. They can
% edit at the same time!
%
% You can upload figures, bibliographies, custom classes and
% styles using the files menu.
%
%%%%%%%%%%%%%%%%%%%%%%%%%%%%%%%%%%%%%%%%%%%%%%%%%%%%%%%%%%%%%%%%%%%%%%

\documentclass[12pt]{article}

\usepackage{sbc-template}
\usepackage{graphicx,url}
\usepackage{float} 
\usepackage[brazil]{babel}
\usepackage{tabularx}
\newcolumntype{C}{>{\centering\arraybackslash}X} % <-- modified
\usepackage[utf8]{inputenc}  
\usepackage[inline]{enumitem}
\usepackage{tcolorbox}


%%%%%%%%%%%%%%%%% Pacotes para inserção de algoritmos - BEGIN
\usepackage[ruled,lined]{algorithm2e}


% Pacote para a definição de novas cores
\usepackage{xcolor}
% Definindo novas cores
\definecolor{verde}{rgb}{0.25,0.5,0.35}
\definecolor{jpurple}{rgb}{0.5,0,0.35}
\definecolor{darkgreen}{rgb}{0.0, 0.2, 0.13}
%\definecolor{oldmauve}{rgb}{0.4, 0.19, 0.28}

\usepackage{listings}
\lstset{
    language=Java,
    basicstyle=\ttfamily\small,
    keywordstyle=\color{jpurple}\bfseries,
    stringstyle=\color{red},
    commentstyle=\color{verde},
    morecomment=[s][\color{blue}]{/**}{*/},
    extendedchars=true,
    showspaces=false,
    showstringspaces=false,
    numbers=left,
    numberstyle=\tiny,
    breaklines=true,
    backgroundcolor=\color{white!10},
    breakautoindent=true,
    captionpos=b,
    xleftmargin=0pt,
    tabsize=2
}
%%%%%%%%%%%%%%%%% Pacotes para inserção de algoritmos - END

     
\sloppy

\title{Adoção e Uso de Virtual Threads em Projetos Open-Source Java: Um Estudo Empírico}

\author{João Paulo de Sales Pimenta\inst{1} }

\address{Instituto de Ciências Exatas e Informática (ICEI) \\ Pontifícia Universidade Católica de Minas Gerais (PUC Minas) \\
Engenharia de Software -- Campus Lourdes (Praça da Liberdade)\\
Rua Alvarenga Peixoto, 159 -- Bairro Lourdes -- Belo Horizonte -- MG -- Brasil\\
CEP: 30180-120
  \email{joao.pimenta.1433569@sga.pucminas.br}
}

\begin{document} 

\maketitle

\begin{abstract}
  This meta-paper describes the style to be used in articles and short papers
  for SBC conferences. For papers in English, you should add just an abstract
  while for the papers in Portuguese, we also ask for an abstract in
  Portuguese (``resumo''). In both cases, abstracts should not have more than
  10 lines and must be in the first page of the paper.
\end{abstract}
     
\begin{resumo} 
  Este meta-artigo descreve o estilo a ser usado na confecção de artigos e
  resumos de artigos para publicação nos anais das conferências organizadas
  pela SBC. É solicitada a escrita de resumo e abstract apenas para os artigos
  escritos em português. Artigos em inglês deverão apresentar apenas abstract.
  Nos dois casos, o autor deve tomar cuidado para que o resumo (e o abstract)
  não ultrapassem 10 linhas cada, sendo que ambos devem estar na primeira
  página do artigo.
\end{resumo}


\begin{tcolorbox}
\footnotesize
\textbf{Bacharelado em Engenharia de \emph{Software} - PUC Minas\\
Trabalho de Conclusão de Curso (TCC)} \\

\indent Orientador de conteúdo (TCC I): Danilo Maia - dqmf88@yahoo.com.br \\
Orientador de conteúdo (TCC I): Leonardo Vilela - leonardocardoso@pucminas.br \\
Orientador de conteúdo (TCC I): Raphael Ramos - rrdcostasi@gmail.com\\
Orientador acadêmico (TCC I): Cleiton Tavares - cleitontavares@pucminas.br \\
Orientador do TCC II: (A ser definido no próximo semestre)\\ \\
Belo Horizonte, DIA de MÊS de ANO.
\end{tcolorbox}


\section{Introdução}
\label{sec:introducao}
O desenvolvimento de software moderno é crescentemente marcado por aplicações distribuídas, principalmente por arquiteturas de microsserviços hospedadas em ambientes de computação em nuvem, nas quais a gestão eficiente de CPU e memória é crítica para desempenho e custo operacional \cite{mohamed:21}. Historicamente, Java aborda concorrência com \textit{platform threads} (threads de plataforma), que encapsulam threads do sistema operacional e, embora familiares, tendem a ser “pesadas” quando mantidas em grande número, especialmente em cargas intensivas de entrada/saída (E/S) \cite{navarro:23, lasic:24}. Modelos assíncronos e reativos difundiram-se nos sistemas para resolver tais limitações por meio de E/S não bloqueante, ao custo de maior complexidade de desenvolvimento e manutenção \cite{navarro:23}. Nesse contexto, o JDK introduziu \textit{virtual threads} (threads virtuais, VT) em \textit{preview} e as estabilizou no JDK~21, conciliando programação síncrona com alta concorrência e baixo custo por tarefa \cite{jep444}.

Apesar desse avanço, \textbf{não há um panorama empírico consolidado sobre a adoção contemporânea de \textit{virtual threads} em projetos Java: onde aparecem, com que prevalência em relação a \textit{threads} de plataforma e quais padrões concretos de uso predominam em diferentes \textit{frameworks} e domínios}. Embora as VTs estejam estáveis no JDK~21 e sua viabilidade técnica tenha sido demonstrada, ainda faltam evidências sistemáticas sobre a influência de fatores como estilo de comunicação entre processos/serviços (IPC, do inglês \textit{Inter-Process Communication}), versões do JDK e bibliotecas de E/S na decisão de adoção, bem como sobre a interação com configurações padrão de \textit{frameworks} e estilos arquiteturais \cite{mohamed:21,navarro:23}.

A relevância do problema decorre das seguintes questões: as \textit{virtual threads} foram concebidas para promover melhoria de desempenho, eficiência e racionalização de recursos em software, otimizando o uso de CPU e memória em aplicações intensivas em \textit{I/O} e aumentando a taxa de transferência (\textit{throughput}) \cite{navarro:23}. Trata-se de uma capacidade estabilizada no JDK~21 \cite{jep444}, sobre a qual ainda há escassez de estudos em cenários realistas, especialmente em aplicações orientadas a banco de dados na nuvem \cite{lasic:24}. Em paralelo, há sinais de adoção em \textit{frameworks} e servidores web populares, como Quarkus, Spring e Jetty \cite{rosa:23}. Nesse contexto, compreender benefícios, custos e desafios de integração torna-se essencial para embasar decisões técnicas, mitigar riscos e orientar a construção de sistemas escaláveis e eficientes. Além disso, a estabilização das VTs no JDK~21 cria um período oportuno para observar tendências reais de adoção e consolidar recomendações que orientem decisões de engenharia com foco em automação, eficiência e uso de recursos \cite{jep444}.

\textbf{O objetivo geral deste trabalho é investigar a adoção contemporânea de \textit{virtual threads} em projetos Java, categorizando sua prevalência, padrões de uso e fatores associados, em contraste com o emprego de \textit{threads} de plataforma}. Especificamente, pretende-se: (i) minerar repositórios de código abertos para identificar indícios de adoção (por exemplo, ocorrências de \texttt{Thread.ofVirtual()} e executores de VT, além de opções de \textit{frameworks} que habilitam VT); (ii) classificar padrões de uso observados (como \textit{thread-per-request}, mapeamentos VTs ou PTs e arranjos híbridos com \textit{pools}); (iii) analisar fatores associados à adoção, incluindo \textit{framework}, versão do JDK, estilo de comunicação entre serviços (APIs \textit{REST} e corretores de mensagens) e bibliotecas de acesso a dados; (iv) examinar a evolução temporal pós-JDK~21 por período, domínio e ecossistema; e (v) identificar barreiras, \textit{smells} e antipadrões documentados em \textit{issues}/\textit{pull requests}, consolidando evidências em uma classificação de adoção.

Como resultados esperados, tem-se: (a) um panorama verificável da prevalência de VTs versus \textit{threads} de plataforma em projetos Java atuais; (b) uma classificação de padrões de uso de VTs com exemplos de código e contexto arquitetural; (c) um conjunto de fatores correlacionados à adoção (ou não adoção) combinando \textit{frameworks}, versões do JDK e bibliotecas de E/S para orientar decisões de engenharia; e (d) um catálogo de barreiras e antipadrões com sugestões de mitigação. Tais artefatos buscam apoiar a automação (simplificação do código concorrente), a eficiência e a racionalização de recursos no desenvolvimento e na operação de software.

Por fim, este trabalho está organizado da seguinte forma: a Seção~2 apresenta o referencial teórico e os trabalhos relacionados sobre VTs em aplicações Java (incluindo conceitos, relatos de adoção e usos característicos); a Seção~3 descreve o desenho do estudo empírico (estratégia de mineração, critérios de inclusão/exclusão, extração de evidências e métricas); a Seção~4 relata os resultados de adoção, padrões e fatores observados; a Seção~5 discute implicações práticas, ameaças à validade e recomendações de engenharia; e a Seção~6 conclui o trabalho e sugere direções futuras.
% \section{Fundamentação Teórica}
\label{sec:fundamentacao}

\subsection{Images}

All images and illustrations should be in black-and-white, or gray tones,
excepting for the papers that will be electronically available (on CD-ROMs,
internet, etc.). The image resolution on paper should be about 600 dpi for
black-and-white images, and 150-300 dpi for grayscale images.  Do not include
images with excessive resolution, as they may take hours to print, without any
visible difference in the result. 



\subsection{Figures and Captions}\label{sec:figs}


Figure and table captions should be centered if less than one line
(Figure~\ref{fig:exampleFig1}), otherwise justified and indented by 0.8cm on
both margins, as shown in Figure~\ref{fig:exampleFig2}. The caption font must
be Helvetica, 10 point, boldface, with 6 points of space before and after each
caption.

\begin{figure}[ht]
\centering
\includegraphics[width=.5\textwidth]{fig1.jpg}
\caption{A typical figure}
\label{fig:exampleFig1}
\end{figure}

\begin{figure}[ht]
\centering
\includegraphics[width=.3\textwidth]{fig2.jpg}
\caption{This figure is an example of a figure caption taking more than one
  line and justified considering margins mentioned in Section~\ref{sec:figs}.}
\label{fig:exampleFig2}
\end{figure}




% \section{Trabalhos Relacionados}
\label{sec:trabalhos_relacionados}

O surgimento das \textit{virtual threads} (VTs) introduziu uma nova abordagem para a concorrência em Java, visando conciliar a simplicidade da programação síncrona com alta escalabilidade e eficiência de recursos, especialmente em cenários intensivos em entrada/saída (E/S). A estabilização dessa capacidade no JDK 21  motiva a investigação de sua adoção contemporânea em projetos \textit{open-source}. Os trabalhos apresentados nesta seção estabelecem a viabilidade técnica e os benefícios de desempenho das VTs em cenários intensivos em E/S, como o acesso a dados e a renderização de \textit{templates} \textit{web}. Estudos demonstram que VTs superam consistentemente as \textit{platform threads} (PTs) tradicionais em eficiência de tempo de execução em tarefas de concorrência e análise de grafos, mas são ligeiramente mais lentas em programação puramente paralela (\textit{CPU-bound}).

La{\v{s}}i{\'c} et al. (2024)\nocite{lasic:24} conduziram um estudo com o objetivo de avaliar a eficiência das \textit{virtual threads} em aplicações de servidor orientadas a banco de dados. O método empregado envolveu a comparação da eficiência das VTs com as PTs tradicionais de Java em aplicações de servidor que utilizam o modelo \textit{thread-per-request} e acessam bancos de dados relacionais (\textit{MySQL, PostgreSQL e Oracle}) e não-relacionais (\textit{MongoDB, Neo4j, Cassandra}). Os resultados preliminares demonstraram que as VTs consistentemente apresentaram desempenho superior ao das \textit{threads} tradicionais, especialmente em operações CRUD de alto \textit{throughput}. Em aplicações com bancos de dados relacionais, a latência foi reduzida em média em aproximadamente $100$ microssegundos. Os autores propõem a utilização de VTs em aplicações de servidor modernas baseadas em \textit{frameworks} e orientadas a banco de dados em nuvem. O estudo  fornece a evidência técnica de que o uso de VTs traz benefícios claros de desempenho em cenários de E/S intensivos, como o acesso a dados, o que é relevante para o objetivo de investigar a adoção empírica das VTs em relação às bibliotecas de acesso a dados.

Outro trabalho realiza uma análise da eficiência das \textit{virtual threads} em cenários de programação paralela (\textit{CPU-bound}). Sirotic et al. (2025) \nocite{sirotic:25} comparou a velocidade de execução de quatro programas paralelos em Java (\textit{Executors, ForkJoin, Streams e Virtual Executors}) para a tarefa de contagem de números primos. O principal resultado encontrado é que as VTs se mostraram ligeiramente mais lentas do que a solução que utilizava \textit{threads} não virtuais para o problema. Por exemplo, a solução com PTs foi marginalmente mais rápida ($1,54$ s \textit{vs.} $1,56$ s) em um processador mais antigo, e ($0,22$ s \textit{vs.} $0,23$ s) em um processador mais novo. Este dado é crucial para o presente estudo, pois o uso de VTs para tarefas puramente \textit{CPU-bound} é considerado um antipadrão, o que, se observado em projetos reais, ajuda a catalogar barreiras e antipadrões. 

Investigando o ambiente de \textit{frameworks} reativos, Navarro et al. (2023)\nocite{navarro:23} analisaram considerações para a integração de \textit{virtual threads} em um \textit{framework} Java reativo (Quarkus), com foco em ambientes com recursos limitados. Os resultados revelaram que a integração inicial de VTs não apresentou desempenho tão bom quanto a abordagem reativa do \textit{framework}. Esta limitação foi atribuída a uma incompatibilidade fundamental entre as VTs e o \textit{framework} Netty, especialmente devido à alta pressão sobre o \textit{Garbage Collector} (GC) causada pelo uso de \textit{ThreadLocals}. O estudo de Navarro \textit{et al.} detalha as complexidades e barreiras técnicas que podem influenciar a adoção de VTs em arquiteturas.

Complementarmente, um estudo explorou como as VTs facilitam o \textit{Progressive Server-Side Rendering} (PSSR) utilizando \textit{template engines} \textit{web} tradicionais que dependem de interfaces bloqueantes. Os autores Pereira et al. (2025)\nocite{pereira:25} concluíram que as VTs permitem que \textit{engines} bloqueantes atinjam escalabilidade comparável àquelas projetadas para E/S não bloqueante, mantendo alto \textit{throughput} e simplificando a implementação de PSSR com estilos de código síncrono familiar. Em implementações Quarkus, a abordagem de VTs alcançou \textit{throughput} elevado ($4856$ requisições por segundo) e escalou efetivamente até $128$ usuários concorrentes. Este trabalho valida que VTs são uma solução eficaz para o padrão de uso \textit{thread-per-request} em aplicações \textit{web} \textit{I/O-bound}, um dos padrões que serão investigados.

Chirila e Sora (2024)\nocite{chirila:24} investigaram a avaliação de desempenho em tempo de execução de \textit{single thread} (ST), PT e VT no contexto da detecção de classes-chave. O objetivo era medir o desempenho da paralelização dos algoritmos \textit{HITS} e \textit{PageRank}, cruciais para a análise de grafos em engenharia de \textit{software}. O método consistiu em executar implementações dos algoritmos usando os três modelos de \textit{threading} (ST, PT e VT) em um conjunto de 14 projetos Java de tamanhos variados. Os algoritmos \textit{HITS} e \textit{PageRank} foram aplicados na representação em grafos dos sistemas analisados, ranqueando as classes por sua importância. O estudo revelou que o modelo VT superou consistentemente os modelos ST e PT, resultando em uma diminuição de tempo de execução de 58,41\% em comparação com o modelo ST no conjunto total de projetos. Eles também observaram que o modelo ST é melhor do que o PT em 12 dos 14 projetos, e o PT só superou o ST em projetos com mais de 1.700 nós, que consomem 90\% do tempo de execução total do conjunto. Este trabalho estabelece um contexto de desempenho geral, demonstrando a maior eficiência das VTs sobre as PTs, fornecendo um forte argumento para a adoção.

Por fim, um trabalho empírico que servirá de base metodológica para a mineração de código proposta é o de Zimmerle et al. (2022)\nocite{zimmerle:22}, que se concentrou na mineração do uso de APIs de \textit{Reactive Programming} (RP) em projetos \textit{open-source}. O objetivo era entender a prevalência do uso dos operadores em três bibliotecas \textit{Reactive Extensions} (Rx) (RxJava, RxJS e RxSwift). O método empregado envolveu a mineração de repositórios no GitHub que possuíam pelo menos $10$ estrelas, utilizado como um filtro de popularidade. Os autores baixaram os projetos e utilizaram expressões regulares (\textit{regex}) para buscar a ocorrência de invocações de operadores. O estudo demonstrou que $95,2\%$ dos operadores Rx estavam em uso, validando a mineração de repositórios como uma abordagem eficaz para quantificar a adoção e o uso de APIs específicas em grande escala. O presente estudo adotará uma abordagem de mineração similar para identificar a adoção de VTs, por exemplo, ocorrências de \textit{Thread.ofVirtual()} e \textit{executors} de VT como instâncias criadas por \textit{Executors.newVirtualThreadPerTaskExecutor()}) em projetos \textit{open-source} Java.
% \section{Materiais e Métodos}
\label{sec:meteriais}



\subsection{Exemplo de Listing}

\lstinputlisting[language=Java,label=alg:code,caption=Code Example]{Code/code.java}


\subsection{Exemplo de Algorithm}

\begin{algorithm}
\begin{scriptsize}
\begin{lstlisting}
void printOwing (double amount){
	print Banner();
	//print details
    System.out.println("name: " + _name);
    System.out.println("amout: " + amount);
}
\end{lstlisting}
\end{scriptsize}

\caption{Código fonte em Java}
\label{alg:codigo_java}

\end{algorithm}




\subsection{Exemplo de Cronograma}

Esta seção apresenta um exemplo de tabela para construção de cronograma.

\begin{table}[!htb]
\caption{Distribuição de tarefas por datas}
\label{tab:datas}
\resizebox{\textwidth}{!}{%
\begin{tabular}{l|ccccccc|}
\cline{2-8}
                                                                                & \multicolumn{7}{c|}{\textbf{2025}}                                                                                                                                              \\ \cline{2-8} 
                                                                                & \multicolumn{2}{c|}{\textbf{fevereiro}}             & \multicolumn{2}{c|}{\textbf{março}}                 & \multicolumn{2}{c|}{\textbf{abril}}                 & \textbf{maio} \\ \hline
\multicolumn{1}{|l|}{\textbf{Tarefas}}                                          & \multicolumn{1}{c|}{1ªQ} & \multicolumn{1}{c|}{2ªQ} & \multicolumn{1}{c|}{1ªQ} & \multicolumn{1}{c|}{2ªQ} & \multicolumn{1}{c|}{1ªQ} & \multicolumn{1}{c|}{2ªQ} & 1ªQ           \\ \hline
\multicolumn{1}{|l|}{Desenvolvimento do \textit{script} para coleta dos repositórios}    & \multicolumn{1}{c|}{X}   & \multicolumn{1}{c|}{}    & \multicolumn{1}{c|}{}    & \multicolumn{1}{c|}{}    & \multicolumn{1}{c|}{}    & \multicolumn{1}{c|}{}    &               \\ \hline
\multicolumn{1}{|l|}{Coleta de dados dos repositórios}                          & \multicolumn{1}{c|}{}    & \multicolumn{1}{c|}{X}   & \multicolumn{1}{c|}{}    & \multicolumn{1}{c|}{}    & \multicolumn{1}{c|}{}    & \multicolumn{1}{c|}{}    &               \\ \hline
\multicolumn{1}{|l|}{\textit{Desenvolvimento da solução}} & \multicolumn{1}{c|}{}    & \multicolumn{1}{c|}{X}   & \multicolumn{1}{c|}{}    & \multicolumn{1}{c|}{}    & \multicolumn{1}{c|}{}    & \multicolumn{1}{c|}{}    &               \\ \hline
\multicolumn{1}{|l|}{Coleta de dados}                                & \multicolumn{1}{c|}{}    & \multicolumn{1}{c|}{}    & \multicolumn{1}{c|}{X}    & \multicolumn{1}{c|}{X}   & \multicolumn{1}{c|}{}   & \multicolumn{1}{c|}{}    &               \\ \hline
\multicolumn{1}{|l|}{Aplicação do Método}                           & \multicolumn{1}{c|}{}    & \multicolumn{1}{c|}{}    & \multicolumn{1}{c|}{X}    & \multicolumn{1}{c|}{X}   & \multicolumn{1}{c|}{X}   & \multicolumn{1}{c|}{}    &               \\ \hline
\multicolumn{1}{|l|}{Geração dos painéis}           & \multicolumn{1}{c|}{}    & \multicolumn{1}{c|}{}    & \multicolumn{1}{c|}{}    & \multicolumn{1}{c|}{}    & \multicolumn{1}{c|}{}    & \multicolumn{1}{c|}{X}   &               \\ \hline
\multicolumn{1}{|l|}{Discussão e avaliação dos resultados}                      & \multicolumn{1}{c|}{}    & \multicolumn{1}{c|}{}    & \multicolumn{1}{c|}{}    & \multicolumn{1}{c|}{}    & \multicolumn{1}{c|}{}    & \multicolumn{1}{c|}{}    & X             \\ \hline
\end{tabular}%
}
\end{table} 



\subsection{References}

Bibliographic references must be unambiguous and uniform.  We recommend giving
the author names references in brackets, e.g. \cite{knuth:84},
\cite{boulic:91}, and \cite{smith:99}.

Cormen et al. (2016)\nocite{cormen2002algoritmos} representa uma citação direta.



The references must be listed using 12 point font size, with 6 points of space
before each reference. The first line of each reference should not be
indented, while the subsequent should be indented by 0.5 cm.

\bibliographystyle{sbc}
\bibliography{sbc-template}

% \appendix
% \section*{Apêndice}
% \section{Exemplo de Seção em um Apêndice}
\label{apendice:secao}

\end{document}
